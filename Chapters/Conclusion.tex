\section{Future Work}

\paragraph{}
There is still work to be completed on this project before future suggestions on other topics or expansions can be made. The tasks that remain to be completed are outlined in the timeline in Section 5. The next major task, as previously discussed, is to use the data provided by QUT as well as floor plans and lighting schematics to model a more feasible design for a commercial building. Once this is completed a more accurate representation of load demand can be analysed. From here specifics of the design can be calculated including current values, voltage drops, cable lengths and locations of devices. Specific devices can be researched and chosen including switchboards, circuit breakers, photovoltaic panels and LED lights. Following calculations and device selection, the financial analysis will be completed as well as an efficiency comparison between a comparable AC system. This is the current overall plan for completing this project. If hurdles or additional ideas arise throughout, it will be adjusted to allow for the highest quality research possible. 

\newpage 

\section{Conclusion}
\paragraph{}
The project being undertaken plans to design and confirm the feasibility of a DC power distribution for commercial buildings to power low load electronics such as lighting and simple devices with a group of photo-voltaic cells. The completion of this task will require extensive research, time, calculations and computer simulations. Milestones that have been set meet the SMART criteria which will allow for tracking and maintaining progress throughout the project. The initial research phase has been completed and designs will begin to be theorised and soon tested following the timeline. 

\paragraph{}
Computer simulations are the main design solution due to the large costs involved in commercial power system implementation. By simulating designs and providing visual aids through 3D rendered images, the presentation will be show not only calculation data but designs implemented on a visual model. In the event that an experimental test can be financially and physically completed and it would benefit the task, it will be done.    

\paragraph{}
Overall it is expected that this research project will be completed with a feasible design. If it is found that no solution will be suitable, a strong justification and possible future areas of discussion will be brought forward. January 2017 will have a complete preliminary design along with justification for the feasibility so that a presentation can be made and successful progress shown. By ensuring that this stage is reached, valuable feedback will be provided via the project supervisor and academic team behind the course. 
\newpage