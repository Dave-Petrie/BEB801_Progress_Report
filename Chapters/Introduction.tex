%%%%%%%%%%%%%%%%%%%%%%%%%%%%%%%%%%%%%%%%%%%%%%%%%%%%%%%%%%%%%%%%%%
%%%%%%%%			INTRODUCTION
%%%%%%%%%%%%%%%%%%%%%%%%%%%%%%%%%%%%%%%%%%%%%%%%%%%%%%%%%%%%%%%%%%

\section{Introduction}

\paragraph{} 
In most Australian homes, power is consumed directly from the local distribution grid. All appliances are connected to one switchboard but can be separated over various circuits with their own protective devices. Generally Australian appliances will use Direct Current (DC) electricity but the outlets provide an Alternating Current (AC) source of 240V at a frequency of 50Hz. Each device therefore requires an inverter that converts the source into the required constant DC voltage and current specific to that device. 

\paragraph{} 
This project will consider the feasibility of converting a portion of power distribution from the standard 240V AC from the grid with an alternative solution. The considered option is utilising a low voltage direct current on a separate grid to power consistently low consumption devices such as lighting or electronics charging devices. There will be both an efficiency and financial analysis completed through hand calculations and software simulations.  

\paragraph{} 
Alternative power generation systems will be considered as well as whether the new possibilities for generation and distribution methods could be used in applications larger than residential homes. The additional locations for this application that will be analysed are apartment and commercial complexes. There will be a variety of design possibilities considered to find the optimal low voltage DC alternative implementation.  

\newpage