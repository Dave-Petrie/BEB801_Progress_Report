%%%%%%%%%%%%%%%%%%%%%%%%%%%%%%%%%%%%%%%%%%%%%%%%%%%%%%%%%%%%%%%%%%
%%%%%%%%			INTRODUCTION
%%%%%%%%%%%%%%%%%%%%%%%%%%%%%%%%%%%%%%%%%%%%%%%%%%%%%%%%%%%%%%%%%%

\section{Introduction}

\paragraph{} 
In most Australian homes, power is consumed directly from the local distribution grid. All appliances are connected to one switchboard but can be separated over various circuits each with their own protective devices, usually circuit breakers. Generally, many modern Australian appliances will use Direct Current (DC) electricity but the outlets provide an Alternating Current (AC) source of 240 \si{V} at a frequency of 50 \si{Hz}. Each device therefore requires an converter that changes the AC source into the required constant DC voltage and current specific to that device. 

\paragraph{} 
This project will consider the feasibility of diverting a portion of power distribution from the standard 240 \si{V AC} sourced from the grid with an alternative solution. The considered option is utilising a low voltage direct current on a separate grid to power known low consumption devices such as LED lighting or electronics charging devices. An efficiency and financial analysis will be completed through hand calculations and software simulations.  

\paragraph{} 
This project will specifically focus on two aspects of this broader topic. These are whether alternative power generation systems will be utilised and whether the new possibilities for generation and distribution methods could be used in applications larger than residential homes. The additional locations for this application that will be analysed are apartment and commercial complexes. There will be a variety of design possibilities considered to find the optimal low voltage DC alternative implementation. To do this there will be a focus on cost, efficiency and usability comparisons of equivalent AC and DC systems.   

\newpage