%%%%%%%%%%%%%%%%%%%%%%%%%%%%%%%%%%%%%%%%%%%%%%%%%%%%%%%%%%%%%%%%%%
%%%%%%%%			PROBLEM DEFINITION
%%%%%%%%%%%%%%%%%%%%%%%%%%%%%%%%%%%%%%%%%%%%%%%%%%%%%%%%%%%%%%%%%%

\section{Research Problems}

\paragraph{}
The key problem that this research paper will be targeting is the feasibility of implementing a separate DC power distribution system for the specific purpose of powering LED lighting circuits and simple electronics. Additionally, the goal is to implement these systems into a commercial building and apartment setting within Australia. For a stronger understanding and case study, Brisbane city will be analysed due to the large amounts of sun and numerous high rises. Designs will be tested through physical devices where possible or software simulations otherwise. In order to answer this key question and complete the project, sub questions were separated and discussed.

\begin{enumerate}
\itemsep-0.5em 
\item Can direct current power be a suitable alternative to alternating current when efficiencies and costs are compared?
\item What is the optimal voltage level for a low voltage DC system when considering loads, costs and efficiencies?
\item If feasible, how can a photo-voltaic system be implemented to power these circuits?
\item Can lighting load and lux requirements be met through this system?
\item If feasible financially and technically, how can the proposed power distribution methods be implemented in commercial buildings effectively?
\end{enumerate} 

\subsection{Current Design Consideration}

\paragraph{}
The research completed and discussions with Geoffrey has allowed for a beginning concept for what could be a feasible design. The main design constraints are cable lengths need to be short, the power generation should be with photo-voltaic systems and due to load constraints, many micro-grids should be used. To do this, with tall and thin buildings it would be possible to use PV cells instead of shading or window awnings to generate electricity. Each floor has their own cells and generates electricity to power their lighting and simple electronics or USB wall charging ports. 

\paragraph{}
Utilising Steven Donohue's findings of 48V being the perfect voltage level for these forms  of systems, the intial plan is to utilise this level \cite{Donohue2014}. The cables being run would follow Australian building standards at $2\text{mm}^{2}$ 2 core and earth and would easily provide the necessary current carrying capacity. These cables would feed to separate, dedicated switchboards for a purely DC supply and then through to LEDs where a highly efficiency DC-DC converter needs to be found or designed. Each floor would therefore have it's own switchboard to power an area between $50\text{m}^{2}$ and $100\text{m}^{2}$ although further calculations are required to check this. An office space is approximately $9\text{m}^{2}$ requiring only 4 LEDs to provide necessary lux levels meaning the load for lighting should not exceed 40 watts and at 48 \si{V DC} that's only 0.83 \si{A}. With multiple rooms such as this, the design should be feasible with further simulations.  

\newpage