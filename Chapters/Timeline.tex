%%%%%%%%%%%%%%%%%%%%%%%%%%%%%%%%%%%%%%%%%%%%%%%%%%%%%%%%%%%%%%%%%%
%%%%%%%%			TIMELINE
%%%%%%%%%%%%%%%%%%%%%%%%%%%%%%%%%%%%%%%%%%%%%%%%%%%%%%%%%%%%%%%%%%
\newpage
\section{Project Timeline}

\paragraph{}
This project will be predominately based on two major resources; software availability and personal time. Software will be available at all times due to the University providing paid software packages and myself already having installed free ones. Personal time will be the major difficulty as it will be balanced between other University classes, part-time work, and personal responsibilities. To balance this specific time periods will be allocated each week to work exclusively in this project.    

\paragraph{}
The tasks were be split into days and University weeks. It was ensured to include the holidays that University allows. This project does not simply end upon the completion of this semester. BEB801 is concluded on November 4th at the end of Week 14, however BEB802 is the subject allocation to complete the second half of this project. The task table will allocate SMART milestones. Additionally, the benefits of completing the subjects during this period is that there is the additional time from summer holidays to account for.

\paragraph{}
Table \ref{table:milestones_1} on the following page shows the milestones of this project. The University assigned submissions are represented as bold text. The four major deliverables for the first half of this project are the library assessment, project proposal, oral presentation and progress report. These four deliverables are what have outlined how the remaining tasks have been created and the time periods allowed for. Earlier due dates are set to allow for editing or possible difficulties to occur without major repercussions. 

\paragraph{}
The assessed deliverables for Semester 2 are similar to the first. The major difference predicted is that the task should be very well understood by the beginning of semester. By having this advantage as well as the additional time during summer break, it allows for a very strong foundation for the final design. This is why the non-assessed milestones are predominately finalisation throughout the entire semester. The summer break will be utilised to reduce the required work during the University period. 

% Table of Tasks
\begin{table}[H]
\centering
\begin{tabular}{||c c||} 
 \hline
 \multicolumn{2}{|c|}{\textbf{Initial Project Timeline}} \\ [0.5ex] 
 \hline\hline
 \textbf{Milestone} & \textbf{Deadline} \\ 
 \hline\hline
 Project Definition & Week 3 \\ 
 \textbf{Library Assessment} & \textbf{Week 4} \\
 Initial Research Phase & Week 6 \\
 \textbf{Project Proposal} & \textbf{Week 7} \\
 Initial Design Phase & Week 9 \\
 Initial Prototype Design Finalised & Week 11 \\
 Initial 3D Modelling for Presentation & Week 12 \\
  \textbf{Initial Oral Presentation} & \textbf{Week 14} \\ 
 \textbf{Written Report} & \textbf{Week 14} \\ 
 Implement Feedback From Report & Summer Break \\
 Complete Research Shortcomings & Summer Break \\
 Complete Further Technical Calculations & Summer Break \\
 Initial Finance Analysis & Week 2 \\
 Design Simulations & Week 6 \\
 \textbf{Progress Report} & \textbf{Week 7} \\
 Finalised Design & Week 10 \\
 Finalised Simulations \& 3D Modelling & Week 11 \\
 Finalised Financial Analysis & Week 12 \\
 \textbf{Final Presentation} & \textbf{Week 14} \\
 \textbf{Final Report} & \textbf{Week 14} \\ [1ex] 
 \hline
\end{tabular}
\caption{Initial Project Timeline}
\label{table:milestones_1}
\end{table}    

\subsection{Stage 1 Analysis of Timeline}

\paragraph{}
This section outlines the initial analysis of the originally projected timeline. Table \ref{table:milestones_2} above will be replicated with additional analysis of whether or not milestones have been reached and if they were on time. Additionally, if aspects of the project has changed and the timeline needs to be re approached, this will be done.  

% Table of Tasks
\begin{table}[H]
\centering
\begin{tabular}{||p{5cm} c c||} 
 \hline
 \multicolumn{3}{|c|}{\textbf{Initial Project Timeline Analysis}} \\ \hline\hline
 \textbf{Milestone} & \textbf{Original Deadline} & \textbf{Actual Completion} \\ [0.5ex] 
 \hline\hline
 Project Definition & Week 3 & Week 3\\ 
 \textbf{Library Assessment} & \textbf{Week 4} & Week 4\\
 Initial Research Phase & Week 6 & Week 6\\
 \textbf{Project Proposal} & \textbf{Week 7} & Week 7\\
 Initial Design Phase & Week 9 & Week 10\\
 Initial Prototype Design Finalised & Week 11 & NA\\
 Initial 3D Modelling for Presentation & Week 12 & Week 11\\ 
 \textbf{Initial Oral Presentation} & \textbf{Week 14} & Week 14\\
 \textbf{Written Report} & \textbf{Week 14} & Week 14\\ 
 Implement Feedback From Report & Summer Break & TBC \\
 Complete Research Shortcomings & Summer Break & TBC \\
 Complete Further Technical Calculations & Summer Break & TBC \\
 Initial Finance Analysis & Week 2 & TBC \\
 Design Simulations & Week 6 & TBC \\ 
 \textbf{Progress Report} & \textbf{Week 7} & TBC \\
 Finalised Design & Week 10 & TBC \\
 Finalised Simulations \& 3D Modelling & Week 11 & TBC \\
 Finalised Financial Analysis & Week 12 & TBC \\ 
 \textbf{Final Presentation} & \textbf{Week 14} & TBC\\
 \textbf{Final Report} & \textbf{Week 14} & TBC\\ [1ex] 
 \hline
\end{tabular}
\caption{Initial Project Timeline Analysis}
\label{table:milestones_2}
\end{table}      

\paragraph{}
The first major change from the first revised timeline is that the initial prototype design is now NA. The reasoning behind this is due to the slight change in planning for the project since the initial proposal submission. Since creation of the timeline, the scope has been re-approached and instead of ensuring a prototype converter is built by the end of the project, the questions have been refocused and more specific simulations will be reached. The design of a converter is a secondary if time allows. 

\paragraph{}
The next milestone was initial 3D modelling of the building design for the presentation. This was completed but not as extensively as I would have liked. The access to Queensland of University's power consumption, floorplans, lighting plans and photovoltaic system configuration. This data was only made available in Week 11 which was later than possibly to fully utilise it before the presentation of progress. Due to this, the summer break tasks will be required to be extended to make up for the loss of time during Semester 1. This data will be used to finalise a floorplan that can be analysed and the concept of a low voltage DC distribution system proven feasibly or not.  

% Table of Tasks
\begin{table}[H]
\centering
\begin{tabular}{||c c||} 
 \hline
 \multicolumn{2}{|c|}{\textbf{Revised Timeline for Remaining Tasks}} \\ \hline\hline
 \textbf{Milestone} & \textbf{Deadline}\\ [0.5ex] 
 \hline\hline
 Implement Feedback From Report & Summer Break  \\
 Complete Research Shortcomings & Summer Break  \\
 Finalise Floorplan and Load Demand & Summer Break  \\
 Complete Further Technical Calculations & Summer Break  \\
 Initial Product Decisions & Week 3  \\
 Initial Finance Analysis & Week 4  \\
 Design Simulations & Week 6  \\ 
 \textbf{Progress Report} & \textbf{Week 7}  \\
 Finalised Design & Week 10  \\
 Finalised Simulations \& 3D Modelling & Week 11  \\
 Finalised Financial Analysis & Week 12  \\ 
 \textbf{Final Presentation} & \textbf{Week 14} \\
 \textbf{Final Report} & \textbf{Week 14} \\ [1ex] 
 \hline
\end{tabular}
\caption{Revised Timeline for Milestones}
\label{table:milestones_3}
\end{table} 
